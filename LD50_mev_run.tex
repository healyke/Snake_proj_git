\documentclass[]{article}
\usepackage{lmodern}
\usepackage{amssymb,amsmath}
\usepackage{ifxetex,ifluatex}
\usepackage{fixltx2e} % provides \textsubscript
\ifnum 0\ifxetex 1\fi\ifluatex 1\fi=0 % if pdftex
  \usepackage[T1]{fontenc}
  \usepackage[utf8]{inputenc}
\else % if luatex or xelatex
  \ifxetex
    \usepackage{mathspec}
  \else
    \usepackage{fontspec}
  \fi
  \defaultfontfeatures{Ligatures=TeX,Scale=MatchLowercase}
\fi
% use upquote if available, for straight quotes in verbatim environments
\IfFileExists{upquote.sty}{\usepackage{upquote}}{}
% use microtype if available
\IfFileExists{microtype.sty}{%
\usepackage{microtype}
\UseMicrotypeSet[protrusion]{basicmath} % disable protrusion for tt fonts
}{}
\usepackage[margin=1in]{geometry}
\usepackage{hyperref}
\hypersetup{unicode=true,
            pdftitle={Snake\_mev\_analysis},
            pdfauthor={Kevin Healy},
            pdfborder={0 0 0},
            breaklinks=true}
\urlstyle{same}  % don't use monospace font for urls
\usepackage{color}
\usepackage{fancyvrb}
\newcommand{\VerbBar}{|}
\newcommand{\VERB}{\Verb[commandchars=\\\{\}]}
\DefineVerbatimEnvironment{Highlighting}{Verbatim}{commandchars=\\\{\}}
% Add ',fontsize=\small' for more characters per line
\usepackage{framed}
\definecolor{shadecolor}{RGB}{248,248,248}
\newenvironment{Shaded}{\begin{snugshade}}{\end{snugshade}}
\newcommand{\KeywordTok}[1]{\textcolor[rgb]{0.13,0.29,0.53}{\textbf{#1}}}
\newcommand{\DataTypeTok}[1]{\textcolor[rgb]{0.13,0.29,0.53}{#1}}
\newcommand{\DecValTok}[1]{\textcolor[rgb]{0.00,0.00,0.81}{#1}}
\newcommand{\BaseNTok}[1]{\textcolor[rgb]{0.00,0.00,0.81}{#1}}
\newcommand{\FloatTok}[1]{\textcolor[rgb]{0.00,0.00,0.81}{#1}}
\newcommand{\ConstantTok}[1]{\textcolor[rgb]{0.00,0.00,0.00}{#1}}
\newcommand{\CharTok}[1]{\textcolor[rgb]{0.31,0.60,0.02}{#1}}
\newcommand{\SpecialCharTok}[1]{\textcolor[rgb]{0.00,0.00,0.00}{#1}}
\newcommand{\StringTok}[1]{\textcolor[rgb]{0.31,0.60,0.02}{#1}}
\newcommand{\VerbatimStringTok}[1]{\textcolor[rgb]{0.31,0.60,0.02}{#1}}
\newcommand{\SpecialStringTok}[1]{\textcolor[rgb]{0.31,0.60,0.02}{#1}}
\newcommand{\ImportTok}[1]{#1}
\newcommand{\CommentTok}[1]{\textcolor[rgb]{0.56,0.35,0.01}{\textit{#1}}}
\newcommand{\DocumentationTok}[1]{\textcolor[rgb]{0.56,0.35,0.01}{\textbf{\textit{#1}}}}
\newcommand{\AnnotationTok}[1]{\textcolor[rgb]{0.56,0.35,0.01}{\textbf{\textit{#1}}}}
\newcommand{\CommentVarTok}[1]{\textcolor[rgb]{0.56,0.35,0.01}{\textbf{\textit{#1}}}}
\newcommand{\OtherTok}[1]{\textcolor[rgb]{0.56,0.35,0.01}{#1}}
\newcommand{\FunctionTok}[1]{\textcolor[rgb]{0.00,0.00,0.00}{#1}}
\newcommand{\VariableTok}[1]{\textcolor[rgb]{0.00,0.00,0.00}{#1}}
\newcommand{\ControlFlowTok}[1]{\textcolor[rgb]{0.13,0.29,0.53}{\textbf{#1}}}
\newcommand{\OperatorTok}[1]{\textcolor[rgb]{0.81,0.36,0.00}{\textbf{#1}}}
\newcommand{\BuiltInTok}[1]{#1}
\newcommand{\ExtensionTok}[1]{#1}
\newcommand{\PreprocessorTok}[1]{\textcolor[rgb]{0.56,0.35,0.01}{\textit{#1}}}
\newcommand{\AttributeTok}[1]{\textcolor[rgb]{0.77,0.63,0.00}{#1}}
\newcommand{\RegionMarkerTok}[1]{#1}
\newcommand{\InformationTok}[1]{\textcolor[rgb]{0.56,0.35,0.01}{\textbf{\textit{#1}}}}
\newcommand{\WarningTok}[1]{\textcolor[rgb]{0.56,0.35,0.01}{\textbf{\textit{#1}}}}
\newcommand{\AlertTok}[1]{\textcolor[rgb]{0.94,0.16,0.16}{#1}}
\newcommand{\ErrorTok}[1]{\textcolor[rgb]{0.64,0.00,0.00}{\textbf{#1}}}
\newcommand{\NormalTok}[1]{#1}
\usepackage{graphicx,grffile}
\makeatletter
\def\maxwidth{\ifdim\Gin@nat@width>\linewidth\linewidth\else\Gin@nat@width\fi}
\def\maxheight{\ifdim\Gin@nat@height>\textheight\textheight\else\Gin@nat@height\fi}
\makeatother
% Scale images if necessary, so that they will not overflow the page
% margins by default, and it is still possible to overwrite the defaults
% using explicit options in \includegraphics[width, height, ...]{}
\setkeys{Gin}{width=\maxwidth,height=\maxheight,keepaspectratio}
\IfFileExists{parskip.sty}{%
\usepackage{parskip}
}{% else
\setlength{\parindent}{0pt}
\setlength{\parskip}{6pt plus 2pt minus 1pt}
}
\setlength{\emergencystretch}{3em}  % prevent overfull lines
\providecommand{\tightlist}{%
  \setlength{\itemsep}{0pt}\setlength{\parskip}{0pt}}
\setcounter{secnumdepth}{0}
% Redefines (sub)paragraphs to behave more like sections
\ifx\paragraph\undefined\else
\let\oldparagraph\paragraph
\renewcommand{\paragraph}[1]{\oldparagraph{#1}\mbox{}}
\fi
\ifx\subparagraph\undefined\else
\let\oldsubparagraph\subparagraph
\renewcommand{\subparagraph}[1]{\oldsubparagraph{#1}\mbox{}}
\fi

%%% Use protect on footnotes to avoid problems with footnotes in titles
\let\rmarkdownfootnote\footnote%
\def\footnote{\protect\rmarkdownfootnote}

%%% Change title format to be more compact
\usepackage{titling}

% Create subtitle command for use in maketitle
\newcommand{\subtitle}[1]{
  \posttitle{
    \begin{center}\large#1\end{center}
    }
}

\setlength{\droptitle}{-2em}
  \title{Snake\_mev\_analysis}
  \pretitle{\vspace{\droptitle}\centering\huge}
  \posttitle{\par}
  \author{Kevin Healy}
  \preauthor{\centering\large\emph}
  \postauthor{\par}
  \predate{\centering\large\emph}
  \postdate{\par}
  \date{05/04/2018}


\begin{document}
\maketitle

\section{Summary}\label{summary}

This script runs the model which includes error associated with the
measurement of LD50 values were this information is reported.

\section{Upload stuff}\label{upload-stuff}

\subsection{Packages}\label{packages}

First the required packages are uploaded. These include .MCMCglmm' which
is used for the main linear analysis, `phytools' and `caper'. to handle
the phylogenies and finally `hdrcde', `wesanderson' and the
`MultiDisPlot.R' function to make some of the plots.

\begin{Shaded}
\begin{Highlighting}[]
\KeywordTok{library}\NormalTok{(phytools)}
\end{Highlighting}
\end{Shaded}

\begin{verbatim}
## Loading required package: ape
\end{verbatim}

\begin{verbatim}
## Loading required package: maps
\end{verbatim}

\begin{Shaded}
\begin{Highlighting}[]
\KeywordTok{library}\NormalTok{(caper)}
\end{Highlighting}
\end{Shaded}

\begin{verbatim}
## Loading required package: MASS
\end{verbatim}

\begin{verbatim}
## Loading required package: mvtnorm
\end{verbatim}

\begin{Shaded}
\begin{Highlighting}[]
\KeywordTok{library}\NormalTok{(MCMCglmm)}
\end{Highlighting}
\end{Shaded}

\begin{verbatim}
## Loading required package: Matrix
\end{verbatim}

\begin{verbatim}
## 
## Attaching package: 'Matrix'
\end{verbatim}

\begin{verbatim}
## The following object is masked from 'package:phytools':
## 
##     expm
\end{verbatim}

\begin{verbatim}
## Loading required package: coda
\end{verbatim}

\begin{Shaded}
\begin{Highlighting}[]
\KeywordTok{library}\NormalTok{(wesanderson)}
\KeywordTok{library}\NormalTok{(hdrcde)}
\end{Highlighting}
\end{Shaded}

\begin{verbatim}
## This is hdrcde 3.2
\end{verbatim}

\begin{Shaded}
\begin{Highlighting}[]
\KeywordTok{source}\NormalTok{(}\StringTok{"MultiDisPlot.R"}\NormalTok{)}
\end{Highlighting}
\end{Shaded}

\subsection{data}\label{data}

Next we upload the main data file. Only species which had a reported
mean yield measure, an LD50 measure and diet with prey item proportions
were included. See the methods section in the main paper for a
description of the data.

\begin{Shaded}
\begin{Highlighting}[]
\NormalTok{ld50_data <-}\StringTok{ }\KeywordTok{read.csv}\NormalTok{(}\StringTok{"S2_snake_data_5_april.csv"}\NormalTok{,}\DataTypeTok{header=}\NormalTok{T,}\DataTypeTok{sep=}\StringTok{","}\NormalTok{)}
\end{Highlighting}
\end{Shaded}

\subsection{Phylogeny}\label{phylogeny}

We use the phylogeny from Pyron et al 2014 to control for phylogeny in
our main model.

\begin{Shaded}
\begin{Highlighting}[]
\NormalTok{tree <-}\StringTok{ }\KeywordTok{read.tree}\NormalTok{(}\StringTok{"liz_and_snake_time_tree.txt"}\NormalTok{)}
\CommentTok{#this seems to fix a dublicate problem caused by polynomies in the tree}
\NormalTok{Tree<-}\KeywordTok{makeLabel}\NormalTok{(tree)}
\NormalTok{Tree<-}\KeywordTok{makeLabel}\NormalTok{(Tree)}
\end{Highlighting}
\end{Shaded}

\subsection{Convert each value into a standard error
term}\label{convert-each-value-into-a-standard-error-term}

We collated measures of error associated with LD50 values when reported.
All error measures associated with LD50 values were converted to
standard error for inclusion as a measurement error term using the
\texttt{mev} term in the \texttt{MCMCglmm} model. Where errors were
reported as confidence intervals (Ci) they were log transformed, to
correspond with the log transformation of the reposne variable, and
converted using \$SE = (u - 95\% lower Ci)/1.96 or \$SE = (u - 99\%
lower Ci)/2.58. For 5\% Fiducial limits we treat these as confidence
intervals and for ranges we treat these as 99\% confidence intervals.
Were Standard deviations are given we use the total number of mice given
as the sample size to first convert it into a 95\% Ci for log
trasnformation and then to SE as above.

\begin{Shaded}
\begin{Highlighting}[]
\NormalTok{se_error <-}\StringTok{ }\KeywordTok{vector}\NormalTok{()}

\ControlFlowTok{for}\NormalTok{(i }\ControlFlowTok{in} \DecValTok{1}\OperatorTok{:}\KeywordTok{length}\NormalTok{(ld50_data}\OperatorTok{$}\NormalTok{ld50_error))\{}
  
   \CommentTok{#95% credability interval}
  \ControlFlowTok{if}\NormalTok{(ld50_data}\OperatorTok{$}\NormalTok{Error_type[i] }\OperatorTok{==}\StringTok{  "95_confidence_interval"}\NormalTok{) }
\NormalTok{    \{se_error[i] <-}\StringTok{  }\NormalTok{((}\KeywordTok{log10}\NormalTok{(ld50_data}\OperatorTok{$}\NormalTok{ld50_mg[i]) }\OperatorTok{-}\StringTok{ }\KeywordTok{log10}\NormalTok{(}\KeywordTok{as.numeric}\NormalTok{(}\KeywordTok{gsub}\NormalTok{(}\StringTok{"-.*"}\NormalTok{, }\StringTok{""}\NormalTok{, ld50_data}\OperatorTok{$}\NormalTok{ld50_error[i])))) }\OperatorTok{/}\StringTok{ }\FloatTok{1.96}\NormalTok{)\}}

  \CommentTok{#Standard deviation will be left as is unless I can get the number of samples}
    \ControlFlowTok{else}\NormalTok{\{  }\ControlFlowTok{if}\NormalTok{(ld50_data}\OperatorTok{$}\NormalTok{Error_type[i] }\OperatorTok{==}\StringTok{  "sd"}\NormalTok{) }
\NormalTok{    \{lower95_temp <-}\StringTok{ }\NormalTok{(ld50_data}\OperatorTok{$}\NormalTok{ld50_mg[i] }\OperatorTok{-}\StringTok{ }\NormalTok{((}\KeywordTok{as.numeric}\NormalTok{(}\KeywordTok{gsub}\NormalTok{(}\StringTok{"-.*"}\NormalTok{, }\StringTok{""}\NormalTok{, ld50_data}\OperatorTok{$}\NormalTok{ld50_error[i])))}\OperatorTok{*}\FloatTok{1.96}\NormalTok{)}\OperatorTok{/}\NormalTok{(}\DecValTok{4}\OperatorTok{^}\FloatTok{0.5}\NormalTok{))}
      
\NormalTok{      se_error[i] <-}\StringTok{  }\NormalTok{((}\KeywordTok{log10}\NormalTok{(ld50_data}\OperatorTok{$}\NormalTok{ld50_mg[i]) }\OperatorTok{-}\StringTok{ }\KeywordTok{log10}\NormalTok{(lower95_temp)) }\OperatorTok{/}\StringTok{ }\FloatTok{1.96}\NormalTok{)\}}
    
  \CommentTok{#for range lets assume a 99% range and a z-value of 2.58}
          \ControlFlowTok{else}\NormalTok{\{  }\ControlFlowTok{if}\NormalTok{(ld50_data}\OperatorTok{$}\NormalTok{Error_type[i] }\OperatorTok{==}\StringTok{  "max_min"} \OperatorTok{|}\StringTok{ }\NormalTok{ld50_data}\OperatorTok{$}\NormalTok{Error_type[i] }\OperatorTok{==}\StringTok{  "range"}\NormalTok{) }
\NormalTok{  \{se_error[i] <-}\StringTok{  }\NormalTok{((}\KeywordTok{log10}\NormalTok{(ld50_data}\OperatorTok{$}\NormalTok{ld50_mg[i]) }\OperatorTok{-}\StringTok{ }\KeywordTok{log10}\NormalTok{(}\KeywordTok{as.numeric}\NormalTok{(}\KeywordTok{gsub}\NormalTok{(}\StringTok{"-.*"}\NormalTok{, }\StringTok{""}\NormalTok{, ld50_data}\OperatorTok{$}\NormalTok{ld50_error[i])))) }\OperatorTok{/}\StringTok{ }\FloatTok{2.58}\NormalTok{)\}}
    
  \CommentTok{#for Fiducial limits we treat them as confidence intervals}
                   \ControlFlowTok{else}\NormalTok{\{  }\ControlFlowTok{if}\NormalTok{(ld50_data}\OperatorTok{$}\NormalTok{Error_type[i] }\OperatorTok{==}\StringTok{  "5% Fiducial limits"}\NormalTok{) }
\NormalTok{  \{se_error[i] <-}\StringTok{  }\NormalTok{((}\KeywordTok{log10}\NormalTok{(ld50_data}\OperatorTok{$}\NormalTok{ld50_mg[i]) }\OperatorTok{-}\StringTok{ }\KeywordTok{log10}\NormalTok{(}\KeywordTok{as.numeric}\NormalTok{(}\KeywordTok{gsub}\NormalTok{(}\StringTok{"-.*"}\NormalTok{, }\StringTok{""}\NormalTok{, ld50_data}\OperatorTok{$}\NormalTok{ld50_error[i])))) }\OperatorTok{/}\StringTok{ }\FloatTok{1.96}\NormalTok{)\}}
       
                     
                                        \ControlFlowTok{else}\NormalTok{\{ se_error[i] <-}\StringTok{ }\OtherTok{NA}
    
                                    
\NormalTok{        \}     }
\NormalTok{      \}}
\NormalTok{    \}}
\NormalTok{  \} }
\NormalTok{\}}
\end{Highlighting}
\end{Shaded}

\begin{verbatim}
## Warning: NAs introduced by coercion
\end{verbatim}

\begin{Shaded}
\begin{Highlighting}[]
\NormalTok{se_error[}\KeywordTok{which}\NormalTok{(}\KeywordTok{is.na}\NormalTok{(se_error))] <-}\StringTok{  }\KeywordTok{max}\NormalTok{(}\KeywordTok{na.omit}\NormalTok{(se_error))}
\end{Highlighting}
\end{Shaded}

\subsection{Data transformation}\label{data-transformation}

\begin{Shaded}
\begin{Highlighting}[]
\NormalTok{ld50_clean  <-}\StringTok{ }\NormalTok{ld50_data}

\CommentTok{#mass}
\NormalTok{mass <-}\StringTok{ }\KeywordTok{log10}\NormalTok{(ld50_clean }\OperatorTok{$}\NormalTok{mass_grams)}

\NormalTok{animal <-}\StringTok{ }\NormalTok{ld50_clean }\OperatorTok{$}\NormalTok{species}
\NormalTok{species <-}\StringTok{ }\NormalTok{ld50_clean }\OperatorTok{$}\NormalTok{species}

\NormalTok{##volume}
\NormalTok{vol.v <-}\StringTok{ }\KeywordTok{log10}\NormalTok{(}\KeywordTok{as.numeric}\NormalTok{(}\KeywordTok{as.vector}\NormalTok{(ld50_clean }\OperatorTok{$}\NormalTok{venom_yield_mg)))}

\CommentTok{#dimension}
\NormalTok{dimensions <-}\StringTok{ }\NormalTok{ld50_clean }\OperatorTok{$}\NormalTok{dimensions}

\CommentTok{#LD50}
\NormalTok{ld50_mg <-}\StringTok{ }\KeywordTok{log10}\NormalTok{(}\KeywordTok{as.vector}\NormalTok{(ld50_clean }\OperatorTok{$}\NormalTok{ld50_mg))}

\NormalTok{ld50_method <-}\StringTok{ }\NormalTok{ld50_clean }\OperatorTok{$}\NormalTok{ld50_method}

\NormalTok{##This just checks if eggs were rcorded within the diet.}
\NormalTok{eggs <-}\StringTok{ }\NormalTok{ld50_clean[,}\KeywordTok{c}\NormalTok{(}\StringTok{"eggs"}\NormalTok{)] }
\NormalTok{egg.bin <-}\StringTok{ }\KeywordTok{rep}\NormalTok{(}\DecValTok{0}\NormalTok{,}\KeywordTok{length}\NormalTok{(eggs))}

\ControlFlowTok{for}\NormalTok{(i }\ControlFlowTok{in} \DecValTok{1}\OperatorTok{:}\NormalTok{(}\KeywordTok{length}\NormalTok{(eggs)))\{}
\ControlFlowTok{if}\NormalTok{(ld50_clean[i,}\KeywordTok{c}\NormalTok{(}\StringTok{"eggs"}\NormalTok{)] }\OperatorTok{>}\StringTok{ }\DecValTok{0}\NormalTok{) egg.bin[i] <-}\StringTok{ "yes"}
\ControlFlowTok{else}\NormalTok{ egg.bin[i] <-}\StringTok{ "no"}
\NormalTok{\}}


\NormalTok{##This is the weigethed evolutionary distance between prey and diet.}
\NormalTok{weigted.dist.v <-}\StringTok{ }\NormalTok{(ld50_clean}\OperatorTok{$}\NormalTok{phylo_distance_diet_model_my)}\OperatorTok{/}\DecValTok{100}

\NormalTok{##the species habitat}
\NormalTok{env <-}\StringTok{ }\NormalTok{ld50_clean}\OperatorTok{$}\NormalTok{environment}

\NormalTok{##whether they constrict}
\NormalTok{con <-}\StringTok{ }\NormalTok{ld50_clean}\OperatorTok{$}\NormalTok{constriction}

\NormalTok{##the taxinomic family of the species}
\NormalTok{fam <-}\StringTok{ }\NormalTok{ld50_clean}\OperatorTok{$}\NormalTok{family}

\NormalTok{##the class of the model used to measure LD50}
\NormalTok{mod_class <-}\StringTok{ }\NormalTok{ld50_clean}\OperatorTok{$}\NormalTok{ld50_model_class}
\end{Highlighting}
\end{Shaded}

\subsection{full model}\label{full-model}

Now we set up the model to insure that there is a corresponding animal
column within the data frame in order for the phylogeny to match up. We
also include as separate ``species'' term to account for variation
associated with multiple measures for a single species.

\begin{Shaded}
\begin{Highlighting}[]
\NormalTok{###full model variables that we want with NA's removed}
\NormalTok{mydat.max <-}\StringTok{ }\KeywordTok{na.omit}\NormalTok{(}\KeywordTok{data.frame}\NormalTok{(animal, }
\NormalTok{                        species, }
\NormalTok{                        mass, }
                        \DataTypeTok{weigted.dist.v =}\NormalTok{ weigted.dist.v, }
                        \DataTypeTok{dimensions =}\NormalTok{ dimensions, }
                        \DataTypeTok{ld50 =}\NormalTok{ ld50_mg, }
\NormalTok{                        ld50_method, }
\NormalTok{                        dimensions, }
\NormalTok{                        egg.bin, }
\NormalTok{                        env, }
\NormalTok{                        con, }
\NormalTok{                        fam,}
\NormalTok{                        mod_class,}
                        \DataTypeTok{se_error =} \KeywordTok{as.vector}\NormalTok{(se_error)))}


\NormalTok{###create a dummy variable of just the species we have to prune the phylogeny}
\NormalTok{spec_dropped <-}\KeywordTok{as.data.frame}\NormalTok{(}\KeywordTok{matrix}\NormalTok{(}\DecValTok{0}\NormalTok{, }\DataTypeTok{nrow =} \KeywordTok{c}\NormalTok{(}\KeywordTok{length}\NormalTok{(}\KeywordTok{unique}\NormalTok{(mydat.max[,}\StringTok{"species"}\NormalTok{]))), }
                                    \DataTypeTok{ncol =} \KeywordTok{c}\NormalTok{(}\DecValTok{2}\NormalTok{), }\DataTypeTok{dimnames =} \KeywordTok{list}\NormalTok{(}\KeywordTok{c}\NormalTok{(),}\KeywordTok{c}\NormalTok{(}\StringTok{"species"}\NormalTok{,}\StringTok{"dummy"}\NormalTok{))))}
\NormalTok{spec_dropped[,}\DecValTok{1}\NormalTok{] <-}\StringTok{ }\KeywordTok{unique}\NormalTok{(mydat.max}\OperatorTok{$}\NormalTok{species)}
\NormalTok{spec_dropped[,}\DecValTok{2}\NormalTok{] <-}\StringTok{ }\KeywordTok{rep}\NormalTok{(}\DecValTok{0}\NormalTok{,}\DataTypeTok{length =} \KeywordTok{c}\NormalTok{(}\KeywordTok{length}\NormalTok{(}\KeywordTok{unique}\NormalTok{(mydat.max[,}\StringTok{"species"}\NormalTok{]))))}
\NormalTok{spec_dropped <-}\StringTok{ }\KeywordTok{as.data.frame}\NormalTok{(spec_dropped)}

\NormalTok{max.dat <-}\StringTok{ }\KeywordTok{comparative.data}\NormalTok{(}\DataTypeTok{data=}\NormalTok{ spec_dropped, }
                            \DataTypeTok{phy=}\NormalTok{ Tree, }
                            \DataTypeTok{names.col=}\StringTok{"species"}\NormalTok{, }
                            \DataTypeTok{vcv=}\OtherTok{FALSE}\NormalTok{)}
\NormalTok{max.tree <-}\StringTok{ }\KeywordTok{chronoMPL}\NormalTok{(max.dat}\OperatorTok{$}\NormalTok{phy)}


\NormalTok{max.dropped <-}\StringTok{ }\NormalTok{max.dat}\OperatorTok{$}\NormalTok{dropped}\OperatorTok{$}\NormalTok{unmatched.rows}

\NormalTok{mydat.MAX <-}\StringTok{ }\NormalTok{mydat.max}
\ControlFlowTok{if}\NormalTok{((}\KeywordTok{length}\NormalTok{(max.dropped) }\OperatorTok{>}\StringTok{ }\KeywordTok{c}\NormalTok{(}\DecValTok{0}\NormalTok{)) }\OperatorTok{==}\StringTok{ }\OtherTok{TRUE}\NormalTok{)\{}
\ControlFlowTok{for}\NormalTok{(i }\ControlFlowTok{in} \DecValTok{1}\OperatorTok{:}\NormalTok{(}\KeywordTok{length}\NormalTok{(max.dropped)))\{}
\NormalTok{mydat.MAX <-mydat.MAX[mydat.MAX}\OperatorTok{$}\NormalTok{species }\OperatorTok{!=}\StringTok{ }\NormalTok{max.dropped[i],]}
\NormalTok{\}}
\NormalTok{\}}


\NormalTok{##clean up the final matched data.frame}
\NormalTok{mydat.MAX <-}\StringTok{ }\KeywordTok{as.data.frame}\NormalTok{(}\KeywordTok{as.matrix}\NormalTok{(mydat.MAX))}

\NormalTok{mydat.MAX <-}\StringTok{ }\KeywordTok{data.frame}\NormalTok{(}\DataTypeTok{animal =}\NormalTok{ mydat.MAX}\OperatorTok{$}\NormalTok{species, }
                        \DataTypeTok{species =}\NormalTok{ mydat.MAX}\OperatorTok{$}\NormalTok{species, }
                        \DataTypeTok{mass.aa =}\NormalTok{ (}\KeywordTok{as.numeric}\NormalTok{(}\KeywordTok{as.character}\NormalTok{(mydat.MAX}\OperatorTok{$}\NormalTok{mass))), }
                        \DataTypeTok{weigted.dist.v =}
\NormalTok{                          (}\KeywordTok{as.numeric}\NormalTok{(}\KeywordTok{as.character}\NormalTok{(mydat.MAX}\OperatorTok{$}\NormalTok{weigted.dist.v))), }
                        \DataTypeTok{ld50 =}\NormalTok{ (}\KeywordTok{as.numeric}\NormalTok{(}\KeywordTok{as.character}\NormalTok{(mydat.MAX}\OperatorTok{$}\NormalTok{ld50))),}
                        \DataTypeTok{ld50_method  =} \KeywordTok{factor}\NormalTok{(mydat.MAX}\OperatorTok{$}\NormalTok{ld50_method, }
                                              \DataTypeTok{levels =} \KeywordTok{c}\NormalTok{(}\StringTok{"sc"}\NormalTok{,}
                                                         \StringTok{"iv"}\NormalTok{,}
                                                         \StringTok{"ip"}\NormalTok{,}
                                                         \StringTok{"im"}\NormalTok{)), }
                        \DataTypeTok{dim =}\NormalTok{ mydat.MAX}\OperatorTok{$}\NormalTok{dimensions, }
                        \DataTypeTok{egg.bin =}\NormalTok{ mydat.MAX}\OperatorTok{$}\NormalTok{egg.bin, }
                        \DataTypeTok{env =} \KeywordTok{factor}\NormalTok{(mydat.MAX}\OperatorTok{$}\NormalTok{env, }
                                     \DataTypeTok{levels =} \KeywordTok{c}\NormalTok{(}\StringTok{"terrestrial"}\NormalTok{,}
                                                \StringTok{"arboreal"}\NormalTok{,}
                                                \StringTok{"aquatic"}\NormalTok{)),}
                        \DataTypeTok{con =}\NormalTok{ mydat.MAX}\OperatorTok{$}\NormalTok{con,}
                        \DataTypeTok{fam =}\NormalTok{ mydat.MAX}\OperatorTok{$}\NormalTok{fam,}
                        \DataTypeTok{mod_class =} \KeywordTok{factor}\NormalTok{(mydat.MAX}\OperatorTok{$}\NormalTok{mod_class, }
                                     \DataTypeTok{levels =} \KeywordTok{c}\NormalTok{(}\StringTok{"mammal"}\NormalTok{,}
                                                \StringTok{"amphibian"}\NormalTok{,}
                                                \StringTok{"artropod"}\NormalTok{,}
                                                \StringTok{"bird"}\NormalTok{,}
                                                \StringTok{"fish"}\NormalTok{,}
                                                \StringTok{"lizard"}\NormalTok{)),}
                        \DataTypeTok{se_er =}\NormalTok{ mydat.MAX}\OperatorTok{$}\NormalTok{se_error)}
\end{Highlighting}
\end{Shaded}

\subsection{Set the parameters}\label{set-the-parameters}

Here we set the number of iterations, the thinning and the burn-in

\begin{Shaded}
\begin{Highlighting}[]
\CommentTok{#if you want to include species and ld50 method as a fixed term.}

\CommentTok{#nitt <- 1200000}
\CommentTok{#thin <- 500}
\CommentTok{#burnin <- 200000}

\NormalTok{nitt <-}\StringTok{ }\DecValTok{12000}
\NormalTok{thin <-}\StringTok{ }\DecValTok{5}
\NormalTok{burnin <-}\StringTok{ }\DecValTok{2000}
\end{Highlighting}
\end{Shaded}

\subsection{Set the prior}\label{set-the-prior}

Set the prior. Here we stick with a uninformative prior.

\begin{Shaded}
\begin{Highlighting}[]
\NormalTok{prior<-}\KeywordTok{list}\NormalTok{(}\DataTypeTok{R =} \KeywordTok{list}\NormalTok{(}\DataTypeTok{V =} \DecValTok{1}\NormalTok{, }\DataTypeTok{nu=}\FloatTok{0.002}\NormalTok{), }
            \DataTypeTok{G =} \KeywordTok{list}\NormalTok{(}\DataTypeTok{G1=}\KeywordTok{list}\NormalTok{(}\DataTypeTok{V =} \DecValTok{1}\NormalTok{,}
                             \DataTypeTok{n =} \DecValTok{1}\NormalTok{, }
                             \DataTypeTok{alpha.mu =} \KeywordTok{rep}\NormalTok{(}\DecValTok{0}\NormalTok{,}\DecValTok{1}\NormalTok{),}
                             \DataTypeTok{alpha.V=} \DecValTok{10}\OperatorTok{^}\DecValTok{3}\NormalTok{), }
                     \DataTypeTok{G1=}\KeywordTok{list}\NormalTok{(}\DataTypeTok{V =} \DecValTok{1}\NormalTok{,}
                             \DataTypeTok{n =} \DecValTok{1}\NormalTok{, }
                             \DataTypeTok{alpha.mu =} \KeywordTok{rep}\NormalTok{(}\DecValTok{0}\NormalTok{,}\DecValTok{1}\NormalTok{), }
                             \DataTypeTok{alpha.V =} \DecValTok{10}\OperatorTok{^}\DecValTok{3}\NormalTok{)))}
\end{Highlighting}
\end{Shaded}

\subsection{Run the full model}\label{run-the-full-model}

Now we run the full model with both venom volume and Ld50 included as
response variables. We also allow for co-variance between these two
terms.

\begin{Shaded}
\begin{Highlighting}[]
\NormalTok{full_model <-}\StringTok{ }\KeywordTok{MCMCglmm}\NormalTok{(ld50 }\OperatorTok{~}\StringTok{ }
\StringTok{                         }\NormalTok{mass.aa }\OperatorTok{+}\StringTok{ }
\StringTok{                         }\NormalTok{ld50_method }\OperatorTok{+}\StringTok{ }
\StringTok{                         }\NormalTok{dim }\OperatorTok{+}\StringTok{ }
\StringTok{                         }\NormalTok{egg.bin }\OperatorTok{+}\StringTok{ }
\StringTok{                         }\NormalTok{weigted.dist.v }\OperatorTok{+}
\StringTok{                         }\NormalTok{mod_class }\OperatorTok{+}
\StringTok{                         }\NormalTok{con }\OperatorTok{+}
\StringTok{                         }\NormalTok{fam, }
                       \DataTypeTok{random =} \OperatorTok{~}\StringTok{ }\NormalTok{animal }\OperatorTok{+}\StringTok{ }
\StringTok{                                 }\NormalTok{species,}
                       \DataTypeTok{rcov =} \OperatorTok{~}\StringTok{ }\NormalTok{units,}
                       \DataTypeTok{mev =}\NormalTok{ mydat.MAX}\OperatorTok{$}\NormalTok{se_error,}
                       \DataTypeTok{data =}\NormalTok{ mydat.MAX, }
                       \DataTypeTok{pedigree =}\NormalTok{ max.tree, }
                       \DataTypeTok{prior =}\NormalTok{ prior,}
                       \DataTypeTok{family =} \KeywordTok{c}\NormalTok{(}\StringTok{"gaussian"}\NormalTok{), }
                       \DataTypeTok{nitt =}\NormalTok{ nitt, }
                       \DataTypeTok{thin =}\NormalTok{ thin, }
                       \DataTypeTok{burnin =}\NormalTok{ burnin,}
                       \DataTypeTok{verbose=} \OtherTok{FALSE}\NormalTok{)}

\KeywordTok{summary}\NormalTok{(full_model)}
\end{Highlighting}
\end{Shaded}

\begin{verbatim}
## 
##  Iterations = 2001:11996
##  Thinning interval  = 5
##  Sample size  = 2000 
## 
##  DIC: 635.6339 
## 
##  G-structure:  ~animal
## 
##        post.mean  l-95% CI u-95% CI eff.samp
## animal    0.3838 4.643e-06   0.7057    120.8
## 
##                ~species
## 
##         post.mean  l-95% CI u-95% CI eff.samp
## species   0.05889 2.066e-09    0.156    76.74
## 
##  R-structure:  ~units
## 
##       post.mean l-95% CI u-95% CI eff.samp
## units    0.1765   0.1534   0.2019     2029
## 
##  Location effects: ld50 ~ mass.aa + ld50_method + dim + egg.bin + weigted.dist.v + mod_class + con + fam 
## 
##                    post.mean  l-95% CI  u-95% CI eff.samp  pMCMC    
## (Intercept)        -0.458513 -1.745286  0.764817     1656  0.463    
## mass.aa             0.258622  0.058325  0.456719     1474  0.012 *  
## ld50_methodiv      -0.380720 -0.505845 -0.261469     2000 <5e-04 ***
## ld50_methodip      -0.290897 -0.416737 -0.176109     1995 <5e-04 ***
## ld50_methodim      -0.048576 -0.209389  0.101691     2000  0.565    
## dim3               -0.314887 -0.665171  0.052405     1256  0.101    
## egg.binyes          1.125189  0.564930  1.649468     2000  0.001 ***
## weigted.dist.v      0.222102  0.147656  0.296963     1238 <5e-04 ***
## mod_classamphibian  0.794413  0.309725  1.218830     2000 <5e-04 ***
## mod_classartropod   2.003471  1.366261  2.647137     2000 <5e-04 ***
## mod_classbird      -0.742039 -0.975972 -0.514100     2610 <5e-04 ***
## mod_classfish      -0.016094 -0.210673  0.178150     2000  0.869    
## mod_classlizard     0.008027 -0.211743  0.217818     2000  0.945    
## conyes             -0.143178 -0.761908  0.417089     1424  0.657    
## famDipsadidae      -0.422792 -1.803625  0.987003     2146  0.546    
## famElapidae        -1.058470 -2.238799  0.240534     2000  0.094 .  
## famLamprophiidae   -0.581534 -2.340967  0.875132     1817  0.464    
## famNatricidae       0.167848 -1.142719  1.544504     2000  0.802    
## famViperidae       -0.309690 -1.723281  1.042709     2000  0.636    
## ---
## Signif. codes:  0 '***' 0.001 '**' 0.01 '*' 0.05 '.' 0.1 ' ' 1
\end{verbatim}

\subsection{Run the full model}\label{run-the-full-model-1}

Now we calculated the proportion of variance associated with phylogeny,
species level and residual for both volume and LD50

\begin{Shaded}
\begin{Highlighting}[]
\NormalTok{ld_50_prop_phlyo <-}\StringTok{ }\NormalTok{full_model}\OperatorTok{$}\NormalTok{VCV[,}\StringTok{"animal"}\NormalTok{]}\OperatorTok{/}
\StringTok{                    }\NormalTok{(full_model}\OperatorTok{$}\NormalTok{VCV[,}\StringTok{"animal"}\NormalTok{] }\OperatorTok{+}
\StringTok{                     }\NormalTok{full_model}\OperatorTok{$}\NormalTok{VCV[,}\StringTok{"species"}\NormalTok{] }\OperatorTok{+}
\StringTok{                     }\NormalTok{full_model}\OperatorTok{$}\NormalTok{VCV[,}\StringTok{"units"}\NormalTok{])}

\NormalTok{ld_50_prop_species <-}\StringTok{ }\NormalTok{full_model}\OperatorTok{$}\NormalTok{VCV[,}\StringTok{"species"}\NormalTok{]}\OperatorTok{/}
\StringTok{                    }\NormalTok{(full_model}\OperatorTok{$}\NormalTok{VCV[,}\StringTok{"animal"}\NormalTok{] }\OperatorTok{+}
\StringTok{                     }\NormalTok{full_model}\OperatorTok{$}\NormalTok{VCV[,}\StringTok{"species"}\NormalTok{] }\OperatorTok{+}
\StringTok{                     }\NormalTok{full_model}\OperatorTok{$}\NormalTok{VCV[,}\StringTok{"units"}\NormalTok{])}

\NormalTok{ld_50_prop_units <-}\StringTok{ }\NormalTok{full_model}\OperatorTok{$}\NormalTok{VCV[,}\StringTok{"units"}\NormalTok{]}\OperatorTok{/}
\StringTok{                    }\NormalTok{(full_model}\OperatorTok{$}\NormalTok{VCV[,}\StringTok{"animal"}\NormalTok{] }\OperatorTok{+}
\StringTok{                     }\NormalTok{full_model}\OperatorTok{$}\NormalTok{VCV[,}\StringTok{"species"}\NormalTok{] }\OperatorTok{+}
\StringTok{                     }\NormalTok{full_model}\OperatorTok{$}\NormalTok{VCV[,}\StringTok{"units"}\NormalTok{])}

\KeywordTok{hdr}\NormalTok{(ld_50_prop_phlyo)}
\end{Highlighting}
\end{Shaded}

\begin{verbatim}
## $hdr
##             [,1]       [,2]      [,3]      [,4]      [,5]      [,6]
## 99% -0.013291345 0.06706244 0.1185889 0.8570325        NA        NA
## 95% -0.004925479 0.05052030 0.1598634 0.1713596 0.2769564 0.8519312
## 50%  0.601386193 0.78185656        NA        NA        NA        NA
## 
## $mode
## [1] 0.7236762
## 
## $falpha
##        1%        5%       50% 
## 0.2335404 0.2957081 2.0484298
\end{verbatim}

\begin{Shaded}
\begin{Highlighting}[]
\KeywordTok{hdr}\NormalTok{(ld_50_prop_species)}
\end{Highlighting}
\end{Shaded}

\begin{verbatim}
## $hdr
##             [,1]      [,2]      [,3]      [,4]     [,5]      [,6]
## 99% -0.015606958 0.4777422 0.4850483 0.5136107       NA        NA
## 95% -0.013596264 0.3300717 0.3392896 0.3622628 0.388692 0.3937955
## 50% -0.005242783 0.0780242        NA        NA       NA        NA
##          [,7]      [,8]
## 99%        NA        NA
## 95% 0.4030342 0.4213126
## 50%        NA        NA
## 
## $mode
## [1] 0.006701967
## 
## $falpha
##        1%        5%       50% 
## 0.2040615 0.4277609 3.7249732
\end{verbatim}

\begin{Shaded}
\begin{Highlighting}[]
\KeywordTok{hdr}\NormalTok{(ld_50_prop_units)}
\end{Highlighting}
\end{Shaded}

\begin{verbatim}
## $hdr
##          [,1]      [,2]
## 99% 0.1375530 0.5359792
## 95% 0.1622049 0.4783413
## 50% 0.2244423 0.3231759
## 
## $mode
## [1] 0.2605166
## 
## $falpha
##        1%        5%       50% 
## 0.2452654 0.8434278 4.1595651
\end{verbatim}

\subsubsection{plot the full model}\label{plot-the-full-model}

We plot the posterior distributions in a table using the
\texttt{MultiDisPlot} function.

\begin{Shaded}
\begin{Highlighting}[]
\NormalTok{ld50_post <-}\StringTok{ }\KeywordTok{list}\NormalTok{(full_model}\OperatorTok{$}\NormalTok{Sol[,}\DecValTok{1}\NormalTok{],}
\NormalTok{                 full_model}\OperatorTok{$}\NormalTok{Sol[,}\StringTok{"mass.aa"}\NormalTok{],}
\NormalTok{                 full_model}\OperatorTok{$}\NormalTok{Sol[,}\StringTok{"ld50_methodiv"}\NormalTok{],}
\NormalTok{                 full_model}\OperatorTok{$}\NormalTok{Sol[,}\StringTok{"ld50_methodip"}\NormalTok{],}
\NormalTok{                 full_model}\OperatorTok{$}\NormalTok{Sol[,}\StringTok{"ld50_methodim"}\NormalTok{],}
\NormalTok{                 full_model}\OperatorTok{$}\NormalTok{Sol[,}\StringTok{"dim3"}\NormalTok{],}
\NormalTok{                 full_model}\OperatorTok{$}\NormalTok{Sol[,}\StringTok{"egg.binyes"}\NormalTok{],}
\NormalTok{                 full_model}\OperatorTok{$}\NormalTok{Sol[,}\StringTok{"weigted.dist.v"}\NormalTok{],}
\NormalTok{                 ld_50_prop_phlyo,}
\NormalTok{                 full_model}\OperatorTok{$}\NormalTok{VCV[,}\StringTok{"animal"}\NormalTok{],}
\NormalTok{                 ld_50_prop_species,}
\NormalTok{                 full_model}\OperatorTok{$}\NormalTok{VCV[,}\StringTok{"species"}\NormalTok{],}
\NormalTok{                 ld_50_prop_units,}
\NormalTok{                 full_model}\OperatorTok{$}\NormalTok{VCV[,}\StringTok{"units"}\NormalTok{]}
\NormalTok{                 )}
\KeywordTok{MultiDisPlot}\NormalTok{(ld50_post)}
\KeywordTok{abline}\NormalTok{(}\DataTypeTok{v =} \DecValTok{0}\NormalTok{, }\DataTypeTok{lty =} \DecValTok{2}\NormalTok{)}
\end{Highlighting}
\end{Shaded}

\includegraphics{LD50_mev_run_files/figure-latex/MultiDisPlot plot-1.pdf}


\end{document}
